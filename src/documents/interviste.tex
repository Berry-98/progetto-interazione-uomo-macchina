\def\people{30}

\section{Interviste}
\subsection{Prefazione}
Questo file riassume i risultati che si sono evinti dalle interviste che abbiamo svolto. Il numero totale di intervistati è di circa \textbf{\people} persone. Abbiamo cercato di intervistare persone che andassero spesso a visitare \textbf{musei e/o monumenti} di \textbf{Roma}.

\subsection{Risultati nel dettaglio}
Andando nel dettaglio, nelle interviste abbiamo toccato, principalmente, i seguenti temi:
    \begin{itemize}
        \item frequenza di visite a musei e/o monumenti;
        \item ricerca di informazioni riguardanti musei e/o monumenti da visitare;
        \item preferenze riguardanti il pagamento dei biglietti;
        \item informazioni riguardanti l'utilizzo o meno delle visite guidate;
        \item informazioni riguardanti sconti universitari (incluse le domeniche gratuite).
    \end{itemize}
    
Grossomodo quasi tutti gli intervistati, come detto nella prefazione, vanno spesso a visitare monumenti e/o musei. Abbiamo chiesto subito come venissero a conoscenza dei luoghi da visitare e abbiamo scoperto che la maggior parte delle persone si informa grazie ai cartelloni pubblicitari esposti in \textbf{metropolitana, autobus e strade}. Inoltre, alcuni ragazzi ci hanno detto che si informano grazie alle stampe presenti sui biglietti della metropolitana. Ovviamente, oltre a questa tipologia di pubblicità, c'è quella che avviene sui \textbf{Social Network}. 
\newline
Per quanto riguarda l'acquisto dei biglietti si è scoperto che molti acquistano i biglietti online in maniera da evitare la fila, invece il restante alla vecchia maniera: cioè al botteghino. In quest'ultimo caso i ragazzi non avevano problemi a fare la fila.
\newline
La maggior parte degli intervistati hanno detto di preferire un'\textbf{audioguida} alle \textbf{visite guidate}. Addirittura in molti hanno ammesso di informarsi autonomamente poiché la visita, altrimenti, sarebbe costata di più. Il restante acquista, senza problemi, la visita guidata.
\newline
Tutti gli intervistati, che sono studenti, utilizzano gli \textbf{sconti universitari}. Riguardo le \textbf{domeniche gratuite}, la maggior parte ha ammesso di voler utilizzare questa offerta ma, a causa del flusso esagerato di persone, ci rinunciano.
Le eventuali recensioni di un museo e/o di un monumento non influenzano  la voglia di visitarlo per la maggior parte delle persone. Tutti gli intervistati hanno ammesso di non lasciare alcuna recensione dopo la visita.
