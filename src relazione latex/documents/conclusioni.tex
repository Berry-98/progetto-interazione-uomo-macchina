\section{Conclusioni}

Durante le fasi di Needfinding e di test del prototipo abbiamo riscontrato un grande interesse nel progetto da parte di svariate persone, ricevendo incoraggiamenti e consigli su nuove funzionalità. Il più delle volte questo interesse è stato però ``deluso'' una volta chiarito che non avremmo realizzato l'applicazione finale.

\paragraph{}
Abbiamo quindi pensato di valutare lo sviluppo vero e proprio dell'app una volta terminato il progetto, in quanto ritenuto fattibile da tutti i membri del gruppo e considerato anche che la maggior parte delle tecnologie necessarie sono già conosciute.

\paragraph{}
Per quanto riguarda il database delle mostre e degli inviti questo potrebbe essere creato semplicemente in \textbf{MySQL}. Per avere una reattività immediata tra applicazione, utente e database si potrebbe utilizzare la tecnologia \textbf{WebSocket}, che è sempre più diffusa. In questa maniera si potrebbero mantenere migliaia di utenti connessi contemporaneamente con un unico server. Per quanto riguarda la parte applicazione vera e propria, si potrebbe utilizzare \textbf{React Native} che combina il \textit{nativo} con \textit{React}. Inoltre, alcuni membri del gruppo hanno già esperienza in progetti che utilizzano \textit{javascript} e \textit{React}, ciò permetterebbe di procedere più velocemente nelle prime fasi dello sviluppo, dovendo evitare di approfondire nuove tecnologie da zero.