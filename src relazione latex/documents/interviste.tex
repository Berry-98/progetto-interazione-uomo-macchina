\def\people{20}

\section{Interviste}

\subsection{Introduzione}
Questa sezione riassume i risultati che si sono evinti dalle interviste che abbiamo svolto. Il numero totale di intervistati è di \textbf{\people} persone. Questa fase è stata divisa in due parti poiché ci siamo resi conto che le prime interviste coprivano un campo troppo generico che abbiamo quindi ristretto. I risultati nel dettaglio si riferiscono alle sole informazioni utili raccolte durante la prima fase e a tutte quelle della seconda.

\paragraph{}
Per rendere più efficiente la fase delle interviste abbiamo deciso di dividerci in due gruppi, uno da 2 persone e l'altro da 3 persone, affinché in ogni gruppo ci fosse almeno una persona che aveva come compito quello di prendere appunti riguardo le risposte dateci dagli intervistati.

\subsection{Risultati nel dettaglio}
Andando nel dettaglio, nelle interviste abbiamo toccato, principalmente, i seguenti temi:
    \begin{itemize}
      \item acquisto biglietti (\textit{solo prima fase});
        \item frequenza di visite a musei;
        \item ricerca di informazioni riguardanti musei da visitare;
        \item informazioni riguardanti l'utilizzo o meno delle visite guidate;
        \item informazioni riguardanti sconti universitari (incluse le domeniche gratuite);
        \item informazioni riguardanti l'utilità delle recensioni.
    \end{itemize}
    
\paragraph{}    
Abbiamo chiesto subito come venissero a conoscenza dei musei da visitare e abbiamo scoperto che la maggior parte delle persone si informa grazie ai cartelloni pubblicitari esposti in \textbf{metropolitana, autobus, strade e volantini}. Ovviamente, oltre a questa tipologia di pubblicità, c'è quella che avviene sui \textbf{Social Network}. Relativamente invece all'acquisto dei biglietti si è scoperto che molti li acquistano online in maniera da evitare la fila, invece il restante alla vecchia maniera: cioè al botteghino. In quest'ultimo caso gli intervistati non avevano problemi ad attendere il proprio turno.

\paragraph{}
La maggior parte degli intervistati ha detto di preferire un'\textbf{audioguida} alle \textbf{visite guidate}. Addirittura in molti hanno ammesso di informarsi autonomamente poiché la visita, altrimenti, sarebbe costata di più.
\newline
Gli \textbf{sconti universitari} vengono utilizzati dalla quasi totalità degli studenti intervistati, riguardo le \textbf{domeniche gratuite} invece, la maggior parte ha ammesso di voler utilizzare questa offerta ma, a causa del flusso esagerato di persone, ci rinunciano nella maggior parte dei casi.

\paragraph{}
Le eventuali recensioni di un museo non influenzano la voglia di visitarlo per la maggior parte delle persone. Tutti gli intervistati hanno ammesso di non lasciare alcuna recensione dopo la visita.


